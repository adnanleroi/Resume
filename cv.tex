%------------------------
% Resume Template
% Author : Adnan AMARA
% Github : https://github.com/adnanleroi
% License : MIT
%------------------------

\documentclass[a4paper,20pt]{article}
\usepackage{latexsym}
\usepackage[empty]{fullpage}
\usepackage{titlesec}
\usepackage{marvosym}
\usepackage[usenames,dvipsnames]{color}
\usepackage{verbatim}
\usepackage{enumitem}
\usepackage[pdftex]{hyperref}
\usepackage{fancyhdr}
\usepackage{tikz}
\usepackage{graphicx,wrapfig}
\pagestyle{fancy}
\fancyhf{} % clear all header and footer fields
\fancyfoot{}
\renewcommand{\headrulewidth}{0pt}
\renewcommand{\footrulewidth}{0pt}

% Adjust margins
\addtolength{\oddsidemargin}{-0.530in}
\addtolength{\evensidemargin}{-0.375in}
\addtolength{\textwidth}{1in}
\addtolength{\topmargin}{-.45in}
\addtolength{\textheight}{1in}

\urlstyle{rm}

\raggedbottom
\raggedright
\setlength{\tabcolsep}{0in}

% Sections formatting
\titleformat{\section}{
  \vspace{-10pt}\scshape\raggedright\large
}{}{0em}{}[\color{black}\titlerule \vspace{-6pt}]

%-------------------------
% Custom commands
\newcommand{\resumeItem}[2]{
  \item\small{
    \textbf{#1}{: #2 \vspace{-2pt}}
  }
}

\newcommand{\resumeItemWithoutTitle}[1]{
  \item\small{
    {\vspace{-2pt}}
  }
}

\newcommand{\resumeSubheading}[4]{
  \vspace{-1pt}\item
    \begin{tabular*}{0.97\textwidth}{l@{\extracolsep{\fill}}r}
      \textbf{#1} & #2 \\
      \textit{#3} & \textit{#4} \\
    \end{tabular*}\vspace{-5pt}
}


\newcommand{\resumeSubItem}[2]{\resumeItem{#1}{#2}\vspace{-3pt}}

\renewcommand{\labelitemii}{$\circ$}

\newcommand{\resumeSubHeadingListStart}{\begin{itemize}[leftmargin=*]}
\newcommand{\resumeSubHeadingListEnd}{\end{itemize}}
\newcommand{\resumeItemListStart}{\begin{itemize}}
\newcommand{\resumeItemListEnd}{\end{itemize}\vspace{-5pt}}

%-----------------------------
%%%%%%  CV STARTS HERE  %%%%%%

\begin{document}

  \begin{wrapfigure}{R}{0.19\textwidth}
         \vspace{-2cm}
        %\begin{center}
        \begin{tikzpicture}
\clip (0,0) circle (2cm) node {\includegraphics[width=4cm]{profile.jpg}};
\end{tikzpicture}
        %\includegraphics[width=0.10\textwidth]{profile.jpg}
        %\end{center}
        % \vspace{-1cm}
 \end{wrapfigure}

%----------HEADING-----------------
\begin{tabular*}{0.80\textwidth}{l@{\extracolsep{\fill}}r}
  \textbf{{\LARGE Adnan Amara}}\\
  \href{https://www.linkedin.com/in/adnan-amara/}{Linkedin: adnan-amara}& Email: \href{mailto:}{adnan.amara@etu.enp-oran.dz}\\
  \href{https://github.com/adnanleroi}{Github: ~~github.com/adnanleroi} & Mobile:~~~+213-792-626-354 \\

  Domicile: xxxxxxxxxxxxxx

\end{tabular*}
\vspace{0.9cm}
%-----------EDUCATION-----------------
\section{~~Éducation et Formation}
  \resumeSubHeadingListStart
    \resumeSubheading
      {École Nationale Polytechnique d'Oran Maurice Audin}{Oran, Algérie}
      {Ingénieur d'état - Électronique et systèmes embarqués}{Decembre 2020 - Juillet 2023}
    %  {\scriptsize \textit{ \footnotesize{\newline{}\textbf{Cours:}Électronique Analogique et numérique, Alimentation pour les systèmes embarqués, Programmation Orienté objet, processeurs avancés, systèmes asservis échantillonnés, Génie logiciel, Circuits et systèmes radio fréquence, Antenne, Intelligence artificiel, robotique, radar,telecommunication.}}}
    \resumeSubheading
      {École Nationale Polytechnique d'Oran Maurice Audin}{Oran, Algérie}
      {Master 2 - Électronique et systèmes embarqués}{Decembre 2020 - Juillet 2023}
      %{\scriptsize \textit{ \footnotesize{\newline{}\textbf{Courses:}: Systèmes Audiovisuels, Traitement d'images, système de communication, fonction d'électronique, électronique de systèmes embarqués,Systèmes d'exploitation, mise en œuvre des systèmes en temps réels , analyse et commande dans l’Espace d’état. }}}
      \resumeSubheading
      {École Nationale Supérieure d'Hydraulique}{Blida, Algérie}
      {Classe préparatoire en Sciences et technologies}{Septembre 2018 - Novembre 2020}
    %  {\scriptsize \textit{ \footnotesize{\newline{}\textbf{Cours:}Analyse, Algèbre , chimie organique, physique, informatique, mécanique rationnelle, mécanique des fluides, électricité générale, résistance des matériaux, analyse numérique, dessin technique, conception assisté par ordinateur, économie.}}}
    \resumeSubheading
      {Lycée Slimane Bouabdellaoui}{Médéa, Algérie}
      {Baccalauréat - Scientifique;  Mention: bien}{July 2018}
    \resumeSubHeadingListEnd
	    
\vspace{-5pt}
\section{Compétances}
	\resumeSubHeadingListStart
	\resumeSubItem{Languages}{~~~~~~Python, C, C++, SQL, VHDL, Octave, Latex, Matlab, Assembleur}
	\resumeSubItem{Frameworks}{~~~~Scikit-learn, TensorFlow, Arduino, Mbed}
    %\resumeSubItem{Librairies}{~~~~~~~~~Pandas, GeoPandas, Numpy, Matplotlib, Seaborn, OpenCV, Selenium, BeautifulSoup, Scrapy, Requests}
	\resumeSubItem{Outils}{~~~~~~~~~~~~~GIT, MySQL,Microsoft Office}
	\resumeSubItem{Platformes}{~~~~~~Linux, Windows, Arduino, Raspberry}
	\resumeSubItem{Soft Skills}{~~~~~~~Leadership, Writing, Public Speaking, Time Management}

\resumeSubHeadingListEnd
\vspace{-5pt}
\section{Expérience}
  \resumeSubHeadingListStart
    
  \resumeSubheading
		{Shariket Kahraba El-Djazair}{Sur site}
		{Stagiaire}{Mars 2021 -  Avril 2021 et Décembre 2021 -  Janvier 2022}
		\resumeItemListStart
        \resumeItem{Role}
          {J'ai la chance de faire 2 stages au niveau de la centale électrique de Berrouaghia SKT, le premier stage était Un stage de découverte de la centrale, le deuxième était un stage plus concret en participant directement aux opérations du service électrique telles que la réparation de machines électriques, l'installation d'armoires électriques, l'analyse des batteries et la connaissance des différents protocoles et systèmes de communication numérique pour l'automatisation des
processus.}
		\resumeItemListEnd
  \resumeSubheading
		{LM electric}{Sur site}
		{Stagiaire}{Juin 2022 - Juillet 2022}
		\resumeItemListStart
        \resumeItem{Role}
          {Dans ce stage, j'étais en charge de concevoir et installer un systèmes d'alimentation électrique en utilisant le logiciel de création des schémas électrique EPLAN et d'automatisme Tia Portal.}
		\resumeItemListEnd

\resumeSubHeadingListEnd

\vspace{-5pt}
\section{expériences bénévoles}
  \resumeSubHeadingListStart
    \resumeSubheading
    {Muhandis Club}{Oran, Algérie}
    {}{Octobre 2022 - Maintenant}
    \resumeItemListStart
    \resumeItem{Role:}
    {Je suis en charge de gérer les réseaux sociaux de ce club scientifique et j'ai l'honneur d'animer le podcast du club Muhandis.}
    \resumeItemListEnd
    
\vspace{5pt}
\resumeSubheading
    {American Corner Oran}{Oran, Algérie}
    {Membre.}{Dec 2020 - Maintenant}
\resumeSubHeadingListEnd

%-----------PROJECTS-----------------
\vspace{-5pt}
\section{Projets}
\resumeSubHeadingListStart
\resumeSubItem{Indicateur de température}{un système dont l'indication tient compte des températures extérieure et intérieure,il possède deux sondes de
température, l'une doit être placée dehors et l'autre à l'intérieur de l'habitat (les deux doivent être à l'ombre).J'ai eu l'idée de construire ce montage suite à des périodes de fortes chaleurs, où l'on sait qu'il vaut mieux aérer la
nuit les pièces de son logement et fermer en journée.}
\vspace{2pt}
\resumeSubItem{Machine CNC basée sur un Arduino UNO}{Il y a tellement de machines CNC dans le monde, dont certaines sont très techniques et complexes à fabriquer ou
même à faire fonctionner correctement. Pour cette raison, j'ai décidé de fabriquer une simple machine CNC basée sur Arduino qui est de loin la plus simple à fabriquer.}
\vspace{2pt}
\resumeSubHeadingListEnd
\vspace{-5pt}
%-----------Awards-----------------
\section{Compétences Linguistiques}
\begin{description}[font=$\bullet$]
\item {Arabe - Langue maternelle}
\vspace{-5pt}
\item {Français - C1}
\vspace{-5pt}
\item {Anglais - C2}
\end{description}

\end{document}
